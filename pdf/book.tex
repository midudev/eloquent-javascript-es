\documentclass[fontsize=13pt,oneside,headings=small,chapterprefix]{scrbook}
\usepackage{listings}
\usepackage{graphicx}
\PassOptionsToPackage{hyphens}{url}\usepackage{hyperref}
\usepackage{natbib}
\usepackage{scrhack}
\usepackage{charter}
\usepackage{bookmark}
\usepackage{ucharclasses}
\usepackage{fontspec}
\usepackage{xcolor}
\usepackage{pdfpages}
\usepackage{arabxetex}
\usepackage{makeidx}

% epigraph is used to style chapter quotes
\usepackage{epigraph}
\setlength{\epigraphwidth}{.8\textwidth}
\setlength{\epigraphrule}{0pt}

\lstset{basicstyle=\ttfamily,xleftmargin=0.8em,breaklines=true,lineskip=-0.2em,aboveskip=0.8em,belowskip=0.8em}
\renewcommand*{\chapterheadstartvskip}{\vspace*{3cm}}
\date{}

\lstset{escapeinside={$<}{>$}}

\makeatletter
\lst@InputCatcodes
\def\lst@DefEC{%
 \lst@CCECUse \lst@ProcessLetter
  ^^80^^81^^82^^83^^84^^85^^86^^87^^88^^89^^8a^^8b^^8c^^8d^^8e^^8f%
  ^^90^^91^^92^^93^^94^^95^^96^^97^^98^^99^^9a^^9b^^9c^^9d^^9e^^9f%
  ^^a0^^a1^^a2^^a3^^a4^^a5^^a6^^a7^^a8^^a9^^aa^^ab^^ac^^ad^^ae^^af%
  ^^b0^^b1^^b2^^b3^^b4^^b5^^b6^^b7^^b8^^b9^^ba^^bb^^bc^^bd^^be^^bf%
  ^^c0^^c1^^c2^^c3^^c4^^c5^^c6^^c7^^c8^^c9^^ca^^cb^^cc^^cd^^ce^^cf%
  ^^d0^^d1^^d2^^d3^^d4^^d5^^d6^^d7^^d8^^d9^^da^^db^^dc^^dd^^de^^df%
  ^^e0^^e1^^e2^^e3^^e4^^e5^^e6^^e7^^e8^^e9^^ea^^eb^^ec^^ed^^ee^^ef%
  ^^f0^^f1^^f2^^f3^^f4^^f5^^f6^^f7^^f8^^f9^^fa^^fb^^fc^^fd^^fe^^ff%
  тявΧαίρετΑΊΡΤΕ“”
  ^^00}
\lst@RestoreCatcodes
\makeatother

\setcounter{secnumdepth}{0}
\setcounter{tocdepth}{1}
\setmonofont[Scale=0.8]{Inconsolata LGC}
\defaultfontfeatures[\emojifont]{Scale=1.15}
\newfontface{\emojifont}{Symbola}
\newfontfamily{\cjkfont}{TW-Sung}
\setTransitionsFor{MiscellaneousSymbolsAndPictographs}{\emojifont}{\normalfont}{}

\newfontfamily{\cinzel}{Cinzel}
\setkomafont{disposition}{\bfseries\cinzel}
\definecolor{silver-chalice}{HTML}{AAAAAA}
\setkomafont{chapterprefix}{\small\color{silver-chalice}}
\RedeclareSectionCommand[innerskip=0pt]{chapter}

\pagestyle{plain}

\usepackage{newunicodechar}
\newunicodechar{π}{$\pi$}
\newunicodechar{ϕ}{$\varphi$}
\newunicodechar{≈}{$\approx$}
\newunicodechar{β}{\ss}
\newunicodechar{⮪}{\emojifont{⮪}}

\graphicspath{{../}}
\definecolor{coveryellow}{rgb}{0.997,0.840,0.122}
\definecolor{blue-bayoux}{rgb}{0.267,0.4,0.467}
\hypersetup{colorlinks,linkcolor=blue-bayoux,urlcolor=black}

\makeindex

\begin{document}

\pagecolor{coveryellow}
\includepdf{../img/cover.jpg}
\pagecolor{white}

\author{Marijn Haverbeke}

\title{Eloquent JavaScript}

\subtitle{Cuarta Edición}

\maketitle

\frontmatter

  \noindent Copyright \textcopyright{} 2024 por Marijn Haverbeke

  \vskip 1em

  \noindent Este trabajo está licenciado bajo la licencia Creative Commons
  attribution-noncommercial
  (\url{http://creativecommons.org/licenses/by-nc/3.0/}). Todo el código de este libro puede ser considerado bajo la licencia MIT
  (\url{https://eloquentjavascript.net/code/LICENSE}).

  Las ilustraciones son contribuciones de varios artistas: Portada por Péchane Sumi-e. Ilustraciones de capítulos por Madalina Tantareanu. Arte de píxeles en los capítulos 7 y 16 por Antonio Perdomo Pastor. Diagramas de expresiones regulares en el capítulo 9 generados con \href{http://regexper.com}{regexper.com} por Jeff Avallone. Fotografía del pueblo en el capítulo 11 por Fabrice Creuzot. Concepto de juego para el capítulo 16 por \href{http://lessmilk.com}{Thomas Palef}.

  Traducción al español por \href{http://twitch.tv/midudev}{midudev}.

  \vskip 1em

  \noindent Puedes comprar una versión impresa de este libro, con un capítulo extra incluido, impreso por No Starch Press en \url{http://a-fwd.com/com=marijhaver-20&asin-com=1593279507}.

{
  \hypersetup{hidelinks}
  \tableofcontents
}

\mainmatter

\input{00_intro.tex}

\input{01_values.tex}

\input{02_program_structure.tex}

\input{03_functions.tex}

\input{04_data.tex}

\input{05_higher_order.tex}

\input{06_object.tex}

\input{07_robot.tex}

\input{08_error.tex}

\input{09_regexp.tex}

\input{10_modules.tex}

\input{11_async.tex}

\input{12_language.tex}

\input{13_browser.tex}

\input{14_dom.tex}

\input{15_event.tex}

\input{16_game.tex}

\input{17_canvas.tex}

\input{18_http.tex}

\input{19_paint.tex}

\input{20_node.tex}

\input{21_skillsharing.tex}

\input{hints.tex}

\backmatter

{
  \hypersetup{hidelinks}
  \printindex
}

\end{document}
